\section{Layout-Empfehlungen für das Board-Design}\label{sec:Layout}

Für ein EMV-konformes Design mit dem LTC3639 sollten folgende Punkte beachtet werden:

\begin{enumerate}
    \item \textbf{Kondensatoren nahe den Pins des LTC3639} platzieren, um parasitäre Induktivitäten und Widerstände zu minimieren.
    \item \textbf{Masseverbindungen} durch ein durchgehendes Masse-Plane sicherstellen, um Rauschen und Spannungsdifferenzen zu reduzieren.
    \item \textbf{Schleifenflächen minimieren}, um elektromagnetische Störungen zu verringern.
    \item \textbf{Hochfrequenz- und Niedrigfrequenzkomponenten} räumlich oder auf verschiedenen Plane-Ebenen trennen.
    \item \textbf{Thermisches Management} durch ausreichende Kühlflächen für Wärmequellen wie den LTC3639.
    \item \textbf{Filterkomponenten} nah an den Regleranschlüssen platzieren, um leitungsgebundene Störungen zu dämpfen.
    \item \textbf{Signalpfade kurz halten} und unnötige Kreuzungen zwischen Hoch- und Niedrigfrequenzleitungen vermeiden.
    \item \textbf{Stromversorgungsleitungen} breit und direkt führen, um Spannungsabfälle und Erwärmung zu minimieren.
\end{enumerate}

Diese Maßnahmen verbessern die EMV-Leistung und die Zuverlässigkeit des Designs.

