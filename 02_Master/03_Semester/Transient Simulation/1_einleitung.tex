\section{Einleitung} \label{sec:einleitung}

Dieser Bericht dokumentiert die Umsetzung der zweiten Aufgabe im Bereich Regelungstechnik und Leistungselektronik. Ziel ist die praxisnahe Anwendung der theoretischen Grundlagen aus den ersten beiden Semestern, indem ein Leistungselektroniksystem modelliert und geregelt wird. Dazu wird ein digitaler Regler entworfen und auf einer geeigneten Plattform implementiert.

Die Aufgabenstellung gliedert sich in zwei wesentliche Arbeitsschritte:
\begin{itemize}
    \item \textbf{Modellierung eines Leistungselektronik-Wandlers in PLECS:}
    Zunächst wird ein Modell eines Leistungselektronik-Wandlers mit PLECS erstellt. Dieses Modell bildet die Basis für die spätere Regelung. Um die Komplexität der Regelung zu minimieren, wird das System so ausgelegt, dass die Dynamik möglichst träge ist. Dadurch kann eine Regelung mit klassischen kontinuierlichen Methoden erfolgen, sofern die Abtastrate hoch genug gewählt wird.  (Siehe Abbildung: \label{fig:Soft})

    \item \textbf{Implementierung des Modells auf einer RTBox mit einem STM-Mikrocontroller als Regler:}
    Nach der Modellierung wird das System auf einer RTBox implementiert. Die Regelung erfolgt über einen STM-Mikrocontroller, der die Steueralgorithmen ausführt und das Systemverhalten in Echtzeit überprüft.
\end{itemize}

Dieser Bericht legt den Fokus auf die präzise Modellbildung, die Auswahl geeigneter Regelparameter und die Implementierung auf der RTBox mit einem STM-Mikrocontroller.