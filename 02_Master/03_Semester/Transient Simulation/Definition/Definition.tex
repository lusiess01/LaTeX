%% -> Deutsche Anpassung 
\usepackage[T1]{fontenc}
\usepackage[ngerman,english]{babel}
%\usepackage{babelbib}
\usepackage{ifthen}
%% --------------------------------------------------------------------- %%

%% -> Farben einfügen
\usepackage{color}
\usepackage{xcolor}

% -> Definierte Farben
\definecolor{MSBlue}{rgb}{.204,.353,.541} 
\definecolor{trueblue}{rgb}{0.0, 0.45, 0.81}
\definecolor{onyx}{rgb}{0.06, 0.06, 0.06}
\definecolor{mygruen}{rgb}{0.4660, 0.6740, 0.1880}
\definecolor{plotblue}{rgb}{0, 0.4470, 0.7410}
\definecolor{myrot}{rgb}{0.8500, 0.3250, 0.0980}
\definecolor{mygreen}{RGB}{28,172,0} %% Matlab Kommentar
\definecolor{mylilas}{RGB}{170,55,241} %% Matlab String

%% --------------------------------------------------------------------- %%

%% -> Grafiken 
\usepackage{graphicx}
\usepackage{epstopdf}
\usepackage{wallpaper}
\usepackage{float} %% Verbessert die Platzierung
\usepackage{placeins} %Für \FloatBarrier beim Einfügen von Abbildungen
\usepackage{matlab-prettifier} % MATLAB code
\usepackage{epstopdf}
%% --------------------------------------------------------------------- %%

\usepackage[tmargin=1in,bmargin=1in,lmargin=1.25in,rmargin=1.25in]{geometry}
\setlength{\parindent}{0em}
\setlength{\parskip}{1.5ex plus0.5ex minus0.5ex}
\usepackage{setspace}
\onehalfspacing

\usepackage{pst-all} %% erweiterte Zeichenbefehle

%% -> Schriftart
\usepackage{helvet}
\renewcommand{\familydefault}{phv}

%% -> Layout Überprüfung
\usepackage{layout}
\usepackage{xspace}

%% -> Für Tabellen
\usepackage{array}
\usepackage{booktabs}
\usepackage{dcolumn}
\usepackage{multirow}

%% -> Einrücken von und Abstand zwischen Absätzen
\setlength{\parindent}{0em}
\setlength{\parskip}{1.5ex plus0.5ex minus0.5ex}

%% -> Tabulator Funktion
\newcommand\tab[1][1cm]{\hspace*{#1}}

%% -> Weniger Warnungen wegen überfüllter Boxen
\tolerance = 9999
\sloppy

%% -> Erweiterte Einstellungen der Bildunterschriften bzw. Tabellenunterschriften
\usepackage{caption}
\usepackage{subcaption}
\captionsetup[figure]{labelfont=footnotesize, textfont=footnotesize}
%\captionsetup[table]{labelfont=footnotesize, textfont=footnotesize}

%% -> Hyperref
%\usepackage[pdftex,bookmarks=true,bookmarksnumbered=true,]{hyperref}

\usepackage[hidelinks]{hyperref}
% \usepackage[colorlinks=true]{hyperref}
% \hypersetup{
% colorlinks=true,
% linkbordercolor={1 0 0},
% citebordercolor={0 1 0},
% filebordercolor={1 0 1},
% runbordercolor={1 0 1},
% urlbordercolor={0 0 1},
% }

%% -> Quellen
\usepackage[super, square, sort&compress]{natbib}

%%-> Weitere nützliche Pakete
\usepackage{todonotes}
\setlength {\marginparwidth }{2cm}
%% --------------------------------------------------------------------- %%

% -> Für Codes zum einfügen
\usepackage{listings}
%\lstset{numbers=left,numberstyle=\tiny,stepnumber=5,numbersep=5pt}

%% -> Für MATLAB Code
\lstset
{language=Matlab,%
    basicstyle=\scriptsize,
    breaklines=true,%
    captionpos=b,
    frame = single,
    morekeywords={matlab2tikz},
    keywordstyle=\color{blue},%
    morekeywords=[2]{1}, keywordstyle=[2]{\color{black}},
    identifierstyle=\color{black},%
    stringstyle=\color{mylilas},
    commentstyle=\color{mygreen},%
    showstringspaces=false,%
    %numbers=left,%
    %numberstyle={\footnotesize \color{black}},%
    %stepnumber=5, %
    %numbersep=9pt, % 
    emph=[1]{for,end,break},emphstyle=[1]\color{red}, %
    emph=[2]{all}, emphstyle=[2]\color{mylilas},    
}
%% --------------------------------------------------------------------- %%

% -> Kopf und Fußzeile gestallten
\usepackage{fancyhdr}
\setlength{\headheight}{13.6pt}
\setlength{\footskip}{41pt}
\rhead{} 
\chead{} 
\lhead{}
\renewcommand{\headrulewidth}{0pt}
\lfoot{\includegraphics[width=\textwidth*1/6]{Definition/DUH-positive.eps}}
\rfoot{\small\thepage}
\cfoot{}
\renewcommand{\footrulewidth}{0pt}
%% --------------------------------------------------------------------- %%

%% -> Matematische Packages
\usepackage{amsmath,amssymb,amsfonts,amstext}
\usepackage{mathtools}
\usepackage{mathrsfs}

%% -> SI-Einheiten
%\usepackage{units}
\usepackage{siunitx}
% \usepackage{physics}
\sisetup{locale = DE,separate-uncertainty}
\sisetup{list-final-separator={ und }}
\sisetup{per-mode=fraction}
\DeclareSIUnit\Pascal{\newton\per\square\metre}
%\sisetup{output-decimal-marker = {.}} %% Für Punkttrennung nur bei englischen Arbeiten erforderlich
%% --------------------------------------------------------------------- %%

%% -> Für die Überschriften
\usepackage{titlesec}
\titleformat{\section}{\LARGE\sffamily\bfseries}{\thesection}{1em}{}
\titleformat{\subsection}{\Large\bfseries\sffamily}{\thesubsection}{1em}{}
\titleformat{\subsubsection}{\large\sffamily\bfseries}{\thesubsubsection}{1em}{}
%% --------------------------------------------------------------------- %%

%% -> Bildnummerierung ist logisch mit Kapitelnummerierung
\numberwithin{figure}{section}
\numberwithin{table}{section}
\numberwithin{equation}{section}
%\numberwithin{equation}{subsection}
%% --------------------------------------------------------------------- %%

%% -> Schaltplan zeichnen und Grafiken und für PGF Plots
\usepackage[european, siunitx]{circuitikz}
\usetikzlibrary{circuits.ee.IEC}
\usepackage{tikz}
\usepackage{tikzscale}
\usepackage{pgfplots}
\pgfplotsset{compat=1.4}
\newlength\figH
\newlength\figW
\setlength{\figH}{6cm}
\setlength{\figW}{8cm}
%% --------------------------------------------------------------------- %%

%% -> Für Vektoren
\usepackage{amsmath}

% For pie-charts
\usepackage{amsmath,amssymb,amsfonts,amstext}
\usepackage{mathtools}
\usepackage{mathrsfs}
\usepackage{pgf-pie}