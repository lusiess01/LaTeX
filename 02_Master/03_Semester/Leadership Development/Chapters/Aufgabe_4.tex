\section{Lego Serious Play}

Die Methode Lego Serious Play (LSP) bietet klare Vorteile, aber auch Nachteile, die je nach Situation berücksichtigt werden müssen.

\subsection{Vorteile der Methode}
\begin{itemize}
    \item \textbf{Visualisierung komplexer Ideen}: LSP ermöglicht es, abstrakte Konzepte durch den Bau von Modellen zu veranschaulichen. Dadurch können Gedanken und Probleme effizienter kommuniziert werden.
    \item \textbf{Förderung von Kreativität und Teamarbeit}: Da jeder Teilnehmer seine Perspektive einbringt, entstehen vielfältige Lösungsansätze, die im Team weiterentwickelt werden.
    \item \textbf{Gleichberechtigung aller Teilnehmer}: Jede Stimme zählt. Die Methode schafft eine Atmosphäre, in der alle Ideen eingebracht und gehört werden, was für eine offene Diskussion sorgt.
\end{itemize}

\subsection{Nachteile der Methode}
\begin{itemize}
    \item \textbf{Zeitaufwand}: Der Prozess erfordert Zeit. Da jeder Teilnehmer seine Modelle erklären muss, kann LSP bei schnelleren Entscheidungsprozessen zu zeitintensiv sein.
    \item \textbf{Begrenzte Tiefe bei komplexen Themen}: Bei sehr technischen oder detaillierten Fragestellungen stößt die Methode an ihre Grenzen.
    \item \textbf{Nicht für jede Gruppe geeignet}: In konservativen oder formellen Umgebungen könnte das spielerische Element auf Ablehnung stoßen und die Effektivität mindern.
\end{itemize}

\subsection{Situative und adaptive Führung}
Durch den Einsatz von LSP habe ich erkannt, wie wichtig es ist, Führung an den Kontext und das Team anzupassen. LSP erfordert ein hohes Maß an Flexibilität, da jede Person ihre eigenen Ideen einbringt. Für mich als Führungskraft bedeutet das, schnell auf verschiedene Kommunikations- und Arbeitsstile reagieren zu können, ohne das gemeinsame Ziel aus den Augen zu verlieren.
Ich habe gelernt, dass adaptive Führung nicht nur darin besteht, Raum für Kreativität zu schaffen, sondern auch die Kontrolle über den Prozess zu behalten. Strukturierte und ergebnisorientierte Führung bleibt der Kern, aber LSP hat mir gezeigt, wann es nötig ist, Kreativität zuzulassen und wann es wichtig ist, wieder klare Vorgaben zu machen.
LSP verdeutlicht, dass situative Führung sowohl das Fördern von Kreativität als auch das Anbieten von Struktur erfordert, um letztlich auf das gemeinsame Ziel hinzuarbeiten.
\newpage