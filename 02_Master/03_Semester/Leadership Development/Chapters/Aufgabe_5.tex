\section{Anwendung der erarbeiteten \glqq einfachen Handlungsrichtlinien\grqq{}}
In der Situation, in der ich ein neues Team übernehme und einer der bestehenden Mitarbeiter öffentlich Unzufriedenheit äußert, würde ich mich auf die zentralen Handlungsrichtlinien konzentrieren, die direkt auf die Herausforderung einzahlen:

\begin{itemize}
    \item \textbf{Offene Tür für die Mitarbeiter haben}: Der erste Schritt wäre ein direktes Gespräch mit dem unzufriedenen Mitarbeiter. Dabei ist es entscheidend, ihm Raum zu geben, seine Bedenken offen zu äußern. Durch aktives Zuhören kann ich seine Perspektive nachvollziehen und gleichzeitig signalisieren, dass ich seine Meinung ernst nehme. Das schafft Vertrauen und verhindert, dass sich der Konflikt weiter zuspitzt.
    \item \textbf{Das Ruder übernehmen und die Mitarbeiter bei sich behalten}: In dieser Phase ist es wichtig, klare Führung zu zeigen. Ich würde dem Team transparent erklären, warum es zu dieser Neuzusammensetzung gekommen ist und welche Ziele damit verfolgt werden. Dadurch stelle ich sicher, dass alle verstehen, dass Veränderungen nicht nur Herausforderungen, sondern auch Chancen mit sich bringen. Meine Rolle ist es, das Team trotz Unstimmigkeiten zusammenzuhalten und den Fokus auf das gemeinsame Ziel zu richten.
    \item \textbf{Brücken zwischen den Mitarbeitern bauen}:Da zwei Mitarbeiter neu im Unternehmen sind und zwei aus anderen Teams kommen, liegt ein klarer Fokus darauf, die Zusammenarbeit zu fördern. Um Konflikten vorzubeugen, würde ich gezielt Maßnahmen ergreifen, um das Team zu integrieren. Dazu könnten Workshops oder kleinere Projekte gehören, bei denen die Mitarbeiter ihre Stärken einbringen und voneinander lernen. Dadurch wird die Zusammenarbeit gefördert und das Team wächst schneller zusammen.
    \item \textbf{Ziele mit den Mitarbeitern zusammen erreichen und sie motivieren}: Um den unzufriedenen Mitarbeiter stärker zu integrieren, würde ich ihm gezielt Verantwortung im Team übertragen. Indem er sieht, dass seine Rolle wichtig ist und sein Beitrag zählt, könnte seine Motivation steigen. Zudem würde ich klar formulieren, welche Ziele wir als Team erreichen wollen, und sicherstellen, dass jeder im Team das Gefühl hat, zu diesem Erfolg beizutragen.
\end{itemize}

Durch diese fokussierte Herangehensweise kann ich sicherstellen, dass der Unmut des Mitarbeiters ernst genommen wird, das Team schnell zusammenwächst und wir gemeinsam an einem Strang ziehen, um unsere Ziele zu erreichen.
\newpage