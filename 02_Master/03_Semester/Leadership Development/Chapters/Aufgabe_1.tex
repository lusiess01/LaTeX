\section{Myers-Briggs-Test}
Im Rahmen meines Myers-Briggs-Ergebnisses, das mich als ESTJ (Exekutive) beschreibt, wird mein Führungsverhalten stark von Struktur, Effizienz und klaren Zielen geprägt. Diese Eigenschaften beeinflussen mein Handeln und meine Entscheidungsfindung, insbesondere in Führungsrollen, positiv. Der ESTJ-Typ legt großen Wert auf bewährte Methoden und etablierte Systeme, was mir dabei hilft, Ordnung und Stabilität in Teams zu bringen.
Im Kontext des LSP-Modells des "Ideal Leaders" zeigen sich deutliche Parallelen zwischen meinen Stärken und bestimmten Führungsaufgaben. So fällt es mir besonders leicht, das Ruder zu übernehmen und die Mitarbeiter bei sich zu behalten, da ich in der Lage bin, schnell Entscheidungen zu treffen und klare Anweisungen zu geben. Meine Fähigkeit, nachhaltige und langfristige Ziele zu überdenken, passt ebenfalls gut zu meinem Pflichtbewusstsein und meinem Fokus auf Organisation und Effizienz. Auch das Erreichen von Zielen gemeinsam mit den Mitarbeitern und deren Motivation werden durch meine direkte, faktenbasierte Kommunikation gefördert, während ich die Firma nach außen hin vertreten und schützen kann, indem ich bestehende Systeme und Strukturen verteidige. Dank meiner organisatorischen Stärken fällt es mir auch leicht, immer alles im Unternehmen im Blick zu behalten, um sicherzustellen, dass Prozesse reibungslos ablaufen.
Jedoch gibt es auch Aufgaben, die mir aufgrund meines Persönlichkeitstyps schwerer fallen könnten. Dazu gehört zum Beispiel, eine offene Tür für die Mitarbeiter zu haben. Mein direkter und pragmatischer Kommunikationsstil könnte dazu führen, dass ich weniger sensibel für die emotionalen Bedürfnisse meiner Mitarbeiter bin. Ähnlich verhält es sich bei der Aufgabe, Brücken zwischen Mitarbeitern zu bauen und zu vernetzen, da ich tendenziell ergebnisorientiert arbeite und weniger flexibel in der Vermittlung zwischen verschiedenen Persönlichkeiten bin. Eine weitere Herausforderung könnte sein, frischen Wind in das Unternehmen zu bringen, da ich eine starke Präferenz für bewährte Methoden habe und neuen, unkonventionellen Ideen gegenüber weniger aufgeschlossen bin.
Insgesamt passen meine Stärken gut zu einer Führungsrolle, die klare Strukturen, Organisation und Effizienz erfordert. Herausforderungen sehe ich vor allem in der emotionalen Führung und der Offenheit für neue, innovative Ideen. Dies sind Bereiche, in denen ich mich weiterentwickeln und mein Führungsverhalten anpassen möchte, um ein noch effektiverer und vielseitigerer Leader zu werden.
\newpage