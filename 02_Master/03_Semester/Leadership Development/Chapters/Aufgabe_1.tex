\section{Myers-Briggs-Test}
Im Rahmen meines Myers-Briggs-Ergebnisses, das mich als ESTJ (Exekutive) beschreibt, zeigt sich, dass mein Führungsverhalten stark von Struktur, Effizienz und klaren Zielen geprägt ist. Diese Kernaspekte meines Persönlichkeitsprofils wirken sich durchweg positiv auf mein Handeln und meine Entscheidungsfindung aus, insbesondere in Führungsrollen. Der ESTJ-Typ legt großen Wert auf bewährte Methoden und etablierte Systeme, was mir hilft, Ordnung und Stabilität in meinen Teams zu schaffen.
Im Kontext des LSP-Modells des  \glqq Ideal Leaders\grqq{} werden diese Stärken besonders deutlich. Meine Fähigkeit, schnell Entscheidungen zu treffen und klare Anweisungen zu geben, erleichtert es mir, das Ruder zu übernehmen und die Mitarbeiter auf den gemeinsamen Erfolg auszurichten. Hierbei spielt meine faktenbasierte, direkte Kommunikation eine zentrale Rolle, um Ziele zu erreichen und meine Teams zu motivieren. Durch meinen organisatorischen Ansatz bin ich zudem in der Lage, die unternehmerischen Abläufe stetig im Blick zu behalten, sodass Prozesse reibungslos verlaufen.
Ein weiteres Merkmal meiner Führungskompetenz ist mein Pflichtbewusstsein. Ich bin stets darauf fokussiert, nachhaltige und langfristige Ziele zu verfolgen, was sich ideal mit meinem Bedürfnis nach Ordnung und Effizienz verbindet. Zudem fällt es mir leicht, die Firma nach außen zu vertreten und die etablierten Strukturen und Systeme zu schützen.
Jedoch gibt es auch Führungsaufgaben, die mir aufgrund meiner Persönlichkeitstypologie weniger leichtfallen. Ein Beispiel ist die emotionale Führung: Mein direkter, pragmatischer Kommunikationsstil birgt das Risiko, weniger sensibel für die emotionalen Bedürfnisse meiner Mitarbeiter zu sein. Darüber hinaus stellt die Vermittlung zwischen verschiedenen Persönlichkeiten innerhalb des Teams eine Herausforderung dar, da ich in der Regel ergebnisorientiert arbeite und weniger flexibel in meiner Herangehensweise bin.
Eine weitere Schwierigkeit besteht darin, frische, innovative Ideen zu fördern. Meine starke Orientierung an bewährten Methoden macht es mir nicht immer leicht, unkonventionellen Ansätzen offen gegenüberzustehen. In dieser Hinsicht sehe ich Potenzial zur Weiterentwicklung.
Insgesamt passen meine Stärken hervorragend zu einer Führungsrolle, die klare Strukturen, Organisation und Effizienz erfordert. Herausforderungen sehe ich in der emotionalen Führung sowie in der Offenheit gegenüber neuen, innovativen Ideen. Diese Bereiche möchte ich gezielt angehen, um mein Führungsverhalten zu verbessern und ein noch effektiverer und vielseitigerer Leader zu werden.
\newpage