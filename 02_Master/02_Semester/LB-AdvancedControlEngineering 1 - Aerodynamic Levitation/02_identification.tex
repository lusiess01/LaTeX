% An explanation of how you identified the system and the choices made in this procedure, e.g. height calculation, step response, identified system etc.
System identification in control engineering is the process of building mathematical models that represent the behavior of real-world systems with analyzing input-output data to infer the dynamics and parameters of the system. We use the data-driven approache which rely on techniques to extract information from measured data. Here the methods like least squares estimation takes place with MATLAB's \textit{lsqcurvefit} function. Of course the accuracy of the identified model depends a lot on the quality of the measured data.

\subsection{PT2 System}
The system behaviour of our current system is figured out by determining the range of the distance sensor in which the system behaves more or less linear. So we choose a range of $20$ to $\SI{60}{cm}$ distance between our table tennis ball and our distance sensor. The already mentioned linearization of the distance sensor can be seen in figure \ref{fig:sensorLinearization} on page \pageref{fig:sensorLinearization}. 

The current system should be identified as a PT2 system, like described in the laboratory instruction. \cite{Laborleitfaden} To do so the system is applied by some input signals to get the output data (see figure \ref{subfig:raw} on page \pageref{subfig:raw}). A nice input/output data timeslow after $\SI{158}{s}$ is chosen to trim the desired system response which is already close to a PT2 system (figure \ref{subfig:trimming} on page \pageref{subfig:trimming}). From here both the input as well as the ouput signal needs to be normalized to $1$. So we divide the input signal by $0.43$ (\textit{max(data.stepTrimmed.y)}) and the output signal by the mean value after settling time. The corresponding normalized plot can be found in figure \ref{subfig:normalize} on page \pageref{subfig:normalize}. From the normalized system response, the fitted PT2 transfer function can be won with \textit{lsqcurvefit} and the response of a damped single-degree-of-freedom harmonic oscillator (eqn. \ref{eqn:oscillator}) which was provided by the lecturer.
\begin{equation}
    y = 1 - \frac{e^{-\zeta w_n t}}{\sqrt{1 - \zeta^2}} \cdot \sin(w_d \cdot t +  tan^{-1} \left(\frac{\sqrt{1 - \zeta^2}}{{\zeta}}\right))
    \label{eqn:oscillator}
\end{equation}
With the \textit{tf2ss} (transfer function to state space) MATLAB function we get the state-space model with the state-space matrices like in equation \ref{eqn:statespacematrices}.
\begin{equation}
\begin{split}
    A &= \begin{bmatrix} -2.20 & -1.57 \cr 1 & 0 \end{bmatrix} \\
    B &= \begin{bmatrix} 1 \cr 0 \end{bmatrix} \\
    C &= \begin{bmatrix} 0 & 1.57 \end{bmatrix} \\
    D &= \begin{bmatrix} 0 \end{bmatrix} \\
    \label{eqn:statespacematrices}
\end{split}
\end{equation}
With a proper state-space representation the controller design as well as the simulink control structure can be started in section \ref{sec:technology}.