In preparation for the laboratory exercise, the system was modeled, and a controller and observer were implemented in Simulink using the provided measurement data. The methodology involved system identification, as detailed in chapter \ref{sec:identification}, where parameters for the State Space Model were computed from the measurement data. These identified parameters were subsequently integrated into the Simulink model (figure \ref{fig:simuSimulation}) utilizing the "Statespace" block (figure \ref{fig:simuModel}).

For the controller, as outlined in chapter \ref{sec:controller}, a State Feedback Control approach was employed, with predefined parameters for percentage overshoot and settling time to ensure precise system control. Additionally, a Luenberger Observer was designed for the modeled system, incorporating considerations of two poles.

In Simulink, the system (figure \ref{fig:simuModel}), the controller (figure \ref{fig:simuController}), and the observer (figure \ref{fig:simuObserver}) were configured as distinct subsystems. This organizational structure not only facilitated better comprehension but also allowed for the facile exchange of the modeled system with input and output blocks for real-world applications.

These simulation and modeling procedures established a methodological framework for the laboratory exercise, as expounded in chapter \ref{sec:identification} and chapter \ref{sec:controller}. The Simulink model, illustrated in figure \ref{fig:simuSimulation}, figure \ref{fig:simuModel}, figure \ref{fig:simuController}, and figure \ref{fig:simuObserver}, served as a comprehensive visual representation of the interconnected components, providing insights into the intricacies of system dynamics. Detailed discussions on system identification, controller design, and observer implementation are available in the referenced chapters, forming the scientific foundation for the conducted simulations and subsequent analyses.