
With meticulous preparation and thorough analysis, significant advancements can be achieved within the confines of this laboratory setting.

Optimization of the PT2 system to the desired specifications was facilitated by strategic adjustments to the positions of both the system poles and observer poles. Particularly noteworthy is the imperative to ensure sensor linearity within a specific operational range, as this significantly influences the attainment of favorable outcomes. A discernible trend observed in the recorded data is the challenge encountered by the controller in maintaining system stability beyond the linearized range of the sensor.

Furthermore, critical considerations were directed towards the setting time and percentage overshoot parameters of the system. These parameters play a pivotal role in expediting system response and controlling overshoot magnitude. It is imperative to strike a harmonious balance among all control parameters to uphold system stability and prevent undesirable phenomena such as oscillations.

In addition to the aforementioned parameters, the impact of disturbances and uncertainties on system performance cannot be overstated. Robust control strategies must be employed to mitigate the effects of external disturbances and uncertainties, ensuring the system's resilience and adaptability in real-world scenarios.

The comprehensive analysis of control strategies and sensor linearization outcomes is encapsulated within the supplementary materials provided in the appendix, serving as a reference for further research and experimentation in the area of control systems engineering.


