% deutsche Anpassungen
%\usepackage[ansinew]{inputenc}
\usepackage[T1]{fontenc}
\usepackage[ngerman,english]{babel}
%\usepackage{babelbib}

% mathematische Symbole
\usepackage{amsmath,amssymb,amsfonts,amstext}

% Listings
\usepackage{listings}
\lstset{numbers=left,numberstyle=\tiny,stepnumber=5,numbersep=5pt}

% erweiterte Zeichenbefehle
\usepackage{pst-all}

% Kopfzeilen frei gestaltbar
\usepackage{fancyhdr}
\lfoot[\fancyplain{}{}]{\fancyplain{}{}}
\rfoot[\fancyplain{}{}]{\fancyplain{}{}}
\cfoot[\fancyplain{}{\footnotesize\thepage}]{\fancyplain{}{\footnotesize\thepage}}
\lhead[\fancyplain{}{\footnotesize\nouppercase\leftmark}]{\fancyplain{}{}}
\chead{}
\rhead[\fancyplain{}{}]{\fancyplain{}{\footnotesize\nouppercase\sc\leftmark}} 

% Farben im Dokument m"oglich
\usepackage{color}

% Schriftart Helvetica
\usepackage{helvet}
\renewcommand{\familydefault}{phv}

% anderdhalbfacher Zeilenabstand
\usepackage{setspace}
\onehalfspacing

% Graphiken einbinden: hier f"ur pdflatex
\usepackage[dvips]{graphicx}

% verbesserte Floating Plazierung
\usepackage{float}

% "Uberpr"ufung des Layouts
\usepackage{layout}

\usepackage{array}

% erweiterte Einstellungen der Bildunterschriften -> 8 Pt
\usepackage[small]{caption}
\captionsetup{belowskip=12pt,aboveskip=4pt}

\usepackage{ifthen}

% H"ohe und Breite des Textk"orpers etwas gr"osser definieren
\usepackage[tmargin=1in,bmargin=1in,lmargin=1.25in,rmargin=1.25in]{geometry}

% Einr"uckung von und Abstand zwischen Abs"atzen
\setlength{\parindent}{0em}
\setlength{\parskip}{1.5ex plus0.5ex minus0.5ex}

% weniger Warnungen wegen "uberf"ullter Boxen
\tolerance = 9999
\sloppy

% Anpassung einiger "Uberschriften 
\renewcommand\figurename{Abbildung}
\renewcommand\tablename{Tabelle}
%\newcommand{\unit}{\mathrm}

% Counter f"ur die Nummerierung
\newcounter{romancount}

% Boolsche Variable f"ur Bachelor-/Masterarbeit oder Bericht
\newboolean{thesis}
