\chapter[Theoretical Background]{Theoretical Background}

\section{Cognitive Rehabilitation: Concepts, Methods, and Target Groups}
Multidisziplinäre Ansätze
\cite{Zucchella.2018}
EEG-Biomarker wie der Brain Symmetry Index (BSI) und der Laterality Coefficient (LC) erlauben eine objektive Bewertung des funktionellen Zustands des Gehirns. Die EEG-Analyse ermöglicht eine individualisierte Rehabilitationssteuerung, indem sie Veränderungen in der Hirnaktivität erfasst – insbesondere im Zusammenhang mit Motor Imagery, einer etablierten kognitiven Rehabilitationsmethode.
Die Zielgruppe der Studie sind Schlaganfallpatienten, die oft sowohl motorische als auch kognitive Beeinträchtigungen aufweisen.



% Campbell Review
\cite{Campbell.2022}
    
\begin{table}[htp]
	\centering
	\caption[Vergleich verschiedener Studien zur taktilen niederfrequenten Vibration in der Demenzbehandlung1]{ergleich verschiedener Studien zur taktilen niederfrequenten Vibration in der Demenzbehandlung}
	\label{tab:TLFV_Demenz}
	\footnotesize
    \resizebox{0.1\linewidth}{!}{
	%\resizebox{.5\textwidth}{!}{
		\begin{tabular}{r c c c l}
			\toprule
			Studie (Autor, Jahr) & Vibrationsart & Dauer / Häufigkeit & Ergebnisse & Anwendungskontext\\
            \midrule
            Clements-Cortes et al., 2016 & Vibroakustisch (40 Hz, Musik, physioakustischer Stuhl) & 2x/Woche, 6 Wochen & Verbesserte SLUMS-Werte, mehr Aufmerksamkeit & Ambulante Einrichtung \\
            Clements-Cortes et al., 2017a & Vibroakustisch (40 Hz, tägliche Heimanwendung) & Täglich, 3 Jahre & Stabile MMSE-Werte über 3 Jahre, reduzierte Frustration & Heimanwendung \\
            Kim und Lee, 2018 & Mechanisch (WBV, Frequenzsteigerung von 20–35 Hz) & 5x/Woche, 8 Wochen & Signifikante EEG-Aktivierung, kognitive Verbesserung & Gemeindezentren \\
            Lam et al., 2018 & Mechanisch (WBV, 30 Hz, 2 mm Amplitude) & 2x/Woche, 9 Wochen & Verbesserte Mobilität, Gleichgewicht, hohe Teilnahmequote & Tagespflege \\
            Heesterbeek et al., 2019a & Mechanisch (WBV, 30 Hz, 1–2 mm Amplitude) & Mehrfach/Woche, Dauer 12 Min & Gute Akzeptanz, einige berichtete Übelkeit & Pflegeheim \\
			\bottomrule
		\end{tabular}
		}
\end{table}

\section{Vibrotactile Stimulation: Principles and Therapeutic Applications}
%Vibrotactile Stimulation
\cite{Campbell.2022}

\cite{ClementsCortes.2016, Heesterbeek.2019,Lam.2018, Clair.1993, Kim.2018, ClementsCortes.2017, Mercado.2006, ClementsCortes.2017b, ClementsCortes.2022}


\section{Actuation Technologies for Haptic Feedbacks}

\section{Voice Coil Actuators for Vibrotactile Stimulation}

\section{Overview of Existing Vibrotactile Stimulation Systems}


[14]–[22] zeigen Wirksamkeit bei AD
%2.3 40 Hz & Gamma Frequenzen	[9], [10], [11], [12] zeigen neurobiologische Wirkung
%2.4 EEG & Wearables	[4]–[8] über EEG-Tech, BCI, und mobile Erfassung
%2.5 VCAs (selbst ergänzt)	Hier kannst du technische Quellen ergänzen, z. B. Datenblätter oder Paper zu haptischen Aktuatoren