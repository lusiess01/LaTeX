\chapter[Theoretical Background]{Theoretical Background}

\section{Neurodegenerative Diseases with Cognitive Impairmentss}
Neurodegenerative diseases such as Alzheimer's disease and Parkinson's disease are among the most common causes of cognitive impairment in old age. They are characterized by the progressive loss of neuronal functions and structures, which in the long term leads to significant impairments in memory, attention, executive functions and coping with everyday life \cite{Harvey.2019}.

\subsection{Alzheimer's Disease}
Alzheimer’s disease is the most common form of dementia, accounting for \SI{60}{\%} to \SI{70}{\%}of all cases \cite{WorldHealthOrganization.2021} According to the Alzheimer's Association (2025), it affects over 55 million people worldwide \cite{Association.2025}. It is characterized by the pathological deposition of proteins such as amyloid-$\boldsymbol{\beta}$-plaques and $\boldsymbol{\tau}$-protein-related neurofibrils in the brain, which leads to the progressive death of nerve cells \cite{Dubois.2016}.

Directly recognizable symptoms and complaints begin with a deterioration in episodic memory, a part of long-term memory that is responsible for remembering personal experiences. As the disease progresses, there are deficits in language, orientation and judgment. The disease is long-lasting and progressive and leads to complete dependence on care in later stages \cite{WorldHealthOrganization.2021}.

\subsection{Parkinson's Disease}
Parkinson's disease is primarily known as a movement disorder, but cognitive symptoms are also common. Around \SI{30}{\%} to \SI{40}{\%} of those affected develop Parkinson's dementia over the course of the disease \cite{Janvin.2005}. Deficits in attention control, executive functions and visual-spatial perception are typical, while memory is often less severely affected at the beginning \cite{Emre.2007}.

The disease occurs because there is too little dopamine in the brain. The reason for this is the death of nerve cells in a certain area of the brain called the \textit{substantia nigra}. In addition, so-called Lewy bodies, which are pathological protein deposits in the nerve cells, can contribute to the development of cognitive symptoms \cite{Kalia.2015}.

\subsection{Cognitive Deficits and Functional Impact}
Cognitive impairments in neurodegenerative diseases affect central functions such as memory, attention, language, visual processing and executive abilities \cite{Harvey.2019}. These deficits have a direct impact on the everyday life of those affected - for example, when taking medication independently, handling money, social contacts or spatial orientation \cite{AmericanPsychiatricAssociation.2013}. In addition to these practical limitations, there are often emotional and psychosocial burdens. Frustration, depressive moods and social isolation are not uncommon and increase the pressure on both those affected and their personal environment.

This makes the early detection of cognitive symptoms all the more important. It forms the basis for targeted interventions and rehabilitation measures that aim to maintain the patient's independence and quality of life for as long as possible \cite{Petersen.2009}.

\section{Cognitive Rehabilitation: Concepts, Methods, and Target Groups}
Multidisziplinäre Ansätze
\cite{Zucchella.2018}

EEG-Biomarker wie der Brain Symmetry Index (BSI) und der Laterality Coefficient (LC) erlauben eine objektive Bewertung des funktionellen Zustands des Gehirns. Die EEG-Analyse ermöglicht eine individualisierte Rehabilitationssteuerung, indem sie Veränderungen in der Hirnaktivität erfasst – insbesondere im Zusammenhang mit Motor Imagery, einer etablierten kognitiven Rehabilitationsmethode.
Die Zielgruppe der Studie sind Schlaganfallpatienten, die oft sowohl motorische als auch kognitive Beeinträchtigungen aufweisen.



% Campbell Review
\cite{Campbell.2022}
    
\begin{table}[htp]
	\centering
	\caption[Vergleich verschiedener Studien zur taktilen niederfrequenten Vibration in der Demenzbehandlung1]{ergleich verschiedener Studien zur taktilen niederfrequenten Vibration in der Demenzbehandlung}
	\label{tab:TLFV_Demenz}
	\footnotesize
    \resizebox{0.1\linewidth}{!}{
	%\resizebox{.5\textwidth}{!}{
		\begin{tabular}{r c c c l}
			\toprule
			Studie (Autor, Jahr) & Vibrationsart & Dauer / Häufigkeit & Ergebnisse & Anwendungskontext\\
            \midrule
            Clements-Cortes et al., 2016 & Vibroakustisch (40 Hz, Musik, physioakustischer Stuhl) & 2x/Woche, 6 Wochen & Verbesserte SLUMS-Werte, mehr Aufmerksamkeit & Ambulante Einrichtung \\
            Clements-Cortes et al., 2017a & Vibroakustisch (40 Hz, tägliche Heimanwendung) & Täglich, 3 Jahre & Stabile MMSE-Werte über 3 Jahre, reduzierte Frustration & Heimanwendung \\
            Kim und Lee, 2018 & Mechanisch (WBV, Frequenzsteigerung von 20–35 Hz) & 5x/Woche, 8 Wochen & Signifikante EEG-Aktivierung, kognitive Verbesserung & Gemeindezentren \\
            Lam et al., 2018 & Mechanisch (WBV, 30 Hz, 2 mm Amplitude) & 2x/Woche, 9 Wochen & Verbesserte Mobilität, Gleichgewicht, hohe Teilnahmequote & Tagespflege \\
            Heesterbeek et al., 2019a & Mechanisch (WBV, 30 Hz, 1–2 mm Amplitude) & Mehrfach/Woche, Dauer 12 Min & Gute Akzeptanz, einige berichtete Übelkeit & Pflegeheim \\
			\bottomrule
		\end{tabular}
		}
\end{table}

\section{Vibrotactile Stimulation: Principles and Therapeutic Applications}
%Vibrotactile Stimulation
\cite{Campbell.2022}

\cite{ClementsCortes.2016, Heesterbeek.2019,Lam.2018, Clair.1993, Kim.2018, ClementsCortes.2017, Mercado.2006, ClementsCortes.2017b, ClementsCortes.2022}


\section{Actuation Technologies for Haptic Feedbacks}
40 Hz \& Gamma Frequenzen, \cite{ReikoTutida.2014} \cite{Mably.2018} \cite{Iaccarino.2016} \cite{Martorell.2019} zeigen neurobiologische Wirkung


\section{Voice Coil Actuators for Vibrotactile Stimulation}
EEG \& Wearables \cite{Guger.2012} \cite{Guger.2017} \cite{SebastianRomagosa.2020} \cite{Xu.2022} über EEG-Tech, BCI, und mobile Erfassung


\section{Overview of Existing Vibrotactile Stimulation Systems}
%VCAs (selbst ergänzt)	Hier kannst du technische Quellen ergänzen, z. B. Datenblätter oder Paper zu haptischen Aktuatoren




