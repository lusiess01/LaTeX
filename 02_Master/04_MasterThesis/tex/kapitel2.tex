\chapter[Theoretical Background]{Theoretical Background}

\section{Cognitive Rehabilitation: Concepts, Methods, and Target Groups}
Multidisziplinäre Ansätze
\cite{Zucchella.2018}

% Campbell Review
\cite{Campbell.2022}

\section{Vibrotactile Stimulation: Principles and Therapeutic Applications}
%Vibrotactile Stimulation
\cite{Campbell.2022}

\cite{ClementsCortes.2016, Heesterbeek.2019,Lam.2018, Clair.1993, Kim.2018, ClementsCortes.2017, Mercado.2006, ClementsCortes.2017b, ClementsCortes.2022}


\section{Actuation Technologies for Haptic Feedbacks}

\section{Voice Coil Actuators for Vibrotactile Stimulation}

\section{Overview of Existing Vibrotactile Stimulation Systems}


[14]–[22] zeigen Wirksamkeit bei AD
%2.3 40 Hz & Gamma Frequenzen	[9], [10], [11], [12] zeigen neurobiologische Wirkung
%2.4 EEG & Wearables	[4]–[8] über EEG-Tech, BCI, und mobile Erfassung
%2.5 VCAs (selbst ergänzt)	Hier kannst du technische Quellen ergänzen, z. B. Datenblätter oder Paper zu haptischen Aktuatoren