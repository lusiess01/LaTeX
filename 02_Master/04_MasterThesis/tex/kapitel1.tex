\chapter[Introduction]{Introduction}

\section{Motivation and Problem Statement}
Dementia is one of the greatest medical and social challenges of the 21st century. More than 55 million people are affected worldwide - and the trend is rising \cite{Association.2025}, \cite{WorldHealthOrganization.2021}. Despite advances in medication, treatment remains limited, particularly in terms of cognitive abilities.

Non-pharmacological interventions are therefore increasingly coming into focus \cite{Zucchella.2018}. Of particular interest are sensory-based approaches such as vibrotactile stimulation, which can activate specific areas of the brain via targeted stimuli. Initial animal studies show positive effects of 40 Hz stimulation on the reduction of amyloid deposits and on cognitive performance \cite{Mably.2018}, \cite{Iaccarino.2016}, \cite{Martorell.2019}.

\section{Objectives of the Thesis}
The aim of this work is to develop a technical system for implementing such stimulation methods. The focus is on a precise, reproducible actuator system that is suitable for applications in the field of cognitive rehabilitation.

\section{Structure of the Thesis}
This work is theoretically and technically oriented and therefore begins with a chapter on the necessary basics. The aim is to provide the medical and technological background knowledge required to understand the system developed.
