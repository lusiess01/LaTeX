\chapter[Introduction]{Introduction}

\section{Motivation and Problem Statement}

%globale Demenzstatistiken (Alzheimer Int., WHO)
\cite{alzint_dementia_statistics}, \cite{who_dementia_factsheet}

%Bedarf an nicht-pharmakologischen Ansätzen
\cite{Zucchella.2018}

%Frühstudien zur 40-Hz-Stimulati (z.B. Gamma-Wellen, Reduktion von Amyloid in Mäusen)
\cite{Mably.2018}, \cite{Iaccarino.2016}, \cite{Martorell.2019}

\section{Objectives of the Thesis}

Erl"autern Sie an dieser Stelle \emph{genau} was ihre Aufgabe ist. Gegebenfalls grenzen Sie auch die Teile aus, welche nicht im Umfang der Arbeit liegen. Dies kann Ihnen gegen Ende ihrer Arbeit bei der Argumentation helfen.

\section{Structure of the Thesis}

Geben Sie in diesem Abschnitt eine grobe Vorausschau auf den Aufbau der Arbeit. Die Arbeit k"onnte empirisch motiviert sein und mit der Auswertung eines Experimentes beginnen oder theoreitsch und somit logischerweise mit einem Theoriekapitel beginnen.
