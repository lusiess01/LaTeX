% General settings (input encoding, font encoding, font, language)
% ****************************************************************************************
\usepackage[utf8]{inputenc} % character encoding used in input file
\usepackage[T1]{fontenc} % specifies the encoding used in the fonts
\usepackage{lmodern} % provides more support for non-ASCII characters than cm-super
\usepackage{microtype} % improves line-filling when using PDFLaTeX
\usepackage[ngerman,english]{babel} % last language is considered the main one
%\usepackage{helvet}
\renewcommand{\familydefault}{\sfdefault} % select a sans serif font family
%
% ****************************************************************************************

% ****************************************************************************************
% Basic packages (graphics, tables, code highlighting, epigraphs)
% ****************************************************************************************
\usepackage{graphicx} % provides enhanced support for including graphics and images
% \usepackage{epstopdf}
\graphicspath{{img/}} % list of directories in which to search for graphics files 
\usepackage[table]{xcolor} % extends the color capabilities of LaTeX
\usepackage{color}
\definecolor{MSBlue}{rgb}{.204,.353,.541} 
\definecolor{trueblue}{rgb}{0.0, 0.45, 0.81}
\definecolor{onyx}{rgb}{0.06, 0.06, 0.06}
\definecolor{mygruen}{rgb}{0.4660, 0.6740, 0.1880}
\definecolor{plotblue}{rgb}{0, 0.4470, 0.7410}
\definecolor{myrot}{rgb}{0.8500, 0.3250, 0.0980}
\definecolor{mygreen}{RGB}{28,172,0} % Matlab comment
\definecolor{mylilas}{RGB}{170,55,241} % Matlab String
\definecolor{MCIBLUE}{RGB}{0,73,131}
\usepackage{tabularx} % extends functionality of the traditional tabular environment
\usepackage{array}
\usepackage{booktabs}
\usepackage{dcolumn}
\usepackage{multirow}

\usepackage{grffile} % improves support for file names with multiple dots
\usepackage{minted} % used to format and highlight programming language source code
% \usepackage{tocloft} % provides means of controlling the typographic design of the ToC
\KOMAoptions{toc=flat} % Glatte Abstände
\KOMAoptions{toc=indent} % Eingerückte Ebenen im Inhaltsverzeichnis
\setlength{\marginparwidth}{2cm}
\usepackage{todonotes}
\usepackage{pst-all} %% erweiterte Zeichenbefehle

% ****************************************************************************************
% Drawing and plotting
% ****************************************************************************************
\usepackage{pgfplots} % provides a high-level interface for creating plots and charts
\pgfplotsset{compat=newest} % sets the compatibility level to the newest version
\usetikzlibrary{plotmarks} % provides various markers (symbols) that can be used
\usetikzlibrary{arrows.meta} % allows customization of arrow tips for paths and arrows
\usepgfplotslibrary{patchplots} % provides additional functionality for handling patches
\usepgfplotslibrary{external} % provides functionality for externalizing plots
\tikzexternalize % activates the externalization feature for TikZ
\newlength\figureheight % declares a new length used to store used to store dimensions
\newlength\figurewidth % declares a new length used to store used to store dimensions
\usepackage{circuitikz} % used for drawing electrical circuits
\usepackage{tikz}
\usepackage{tikzscale}

% ****************************************************************************************
% Mathematics and physics
% ****************************************************************************************
\usepackage{amsmath,amssymb,amsfonts,amstext}
\usepackage{mathtools}
\usepackage{mathrsfs}

%% -> SI-Einheiten
\usepackage{siunitx}
% Einstellungen für SI-Einheiten
\sisetup{
	locale = DE,                        % Deutsche Einstellungen
	output-decimal-marker = {,},        % Dezimaltrennzeichen auf "," setzen
	exponent-product = \cdot,           % Exponenten mit "·" trennen
	per-mode = symbol-or-fraction,      % "pro" als Symbol oder Potenz
	parse-numbers = true,               % Automatische Nummerninterpretation
	bracket-numbers = false             % Entfernt Klammern bei negativen Exponenten
}

% ****************************************************************************************
% Referencing and citing
% ****************************************************************************************
\usepackage[
	format=plain,% typeset caption as normal paragraph
	labelformat=simple,% typeset label as name and number
	labelsep=period,% caption label and text separated by period and space
	textformat=simple,% caption text typeset as is (without additional formatting)
	justification=justified,% sets the justification of the caption text to be justified
	singlelinecheck=true,% automatically center short captions
	font=small,% set font size to small
	labelfont=bf,% set bold font for label
	width=.75\textwidth% set fixed width for caption
]{caption} % customize the formatting of captions
\usepackage{subcaption}
\captionsetup[figure]{
	labelfont=bf,                   % Beschriftung für Abbildungen fett
	textfont=normal,                % Text normal
	labelsep=space,                 % Punkt nach der Beschriftung
	%belowskip=10pt
}
\captionsetup[table]{
	labelfont=bf,                   % Beschriftung für Tabellen fett
	textfont=normal,                % Text normal
	labelsep=space,                 % Punkt nach der Beschriftung
	belowskip=10pt
}
% Titel von Abbildungen und Tabellen ändern
\renewcommand\figurename{Abbildung}
\renewcommand\tablename{Tabelle}
% ****************************************************************************************

% \usepackage{hyperref} % hypertext marks (should be loaded last but before geometry)
% \hypersetup{
%     colorlinks,% colours the text of links and anchors (instead of borders)
% 	linkcolor={MCIBLUE},%
% 	citecolor={MCIBLUE},%
% 	urlcolor={MCIBLUE},%
%   % pdftitle={\thesisTitle},% PDF display and information options
% 	% pdfsubject={\thesisType},%
% 	% pdfauthor={\thesisStudent},%
% 	pdfkeywords={thesis},%
% 	pdfcreator={pdflatex},%
% 	pdfproducer={LaTeX with hyperref}%
% }
\usepackage[
bookmarks=true,
bookmarksopen=true,
bookmarksnumbered=true,
pdfborder={0 0 0},
colorlinks=true,
linkcolor=MCIBLUE,
citecolor=MCIBLUE,
urlcolor=MCIBLUE
]{hyperref}
\usepackage[capitalise]{cleveref} % enhances cross-referencing features
\crefformat{equation}{(#2#1#3)}
\Crefformat{equation}{Equation~(#2#1#3)}
\crefname{figure}{Abbildung}{Abbildungen} % Singular und Plural für Abbildungen
\crefname{table}{Tabelle}{Tabellen}       % Singular und Plural für Tabellen
\crefname{equation}{Gleichung}{Gleichungen} % Singular und Plural für Gleichungen

% \Crefname{figure}{Abbildung}{Abbildungen} % Für den Satzbeginn mit großem Anfangsbuchstaben
% \Crefname{table}{Tabelle}{Tabellen}
% \Crefname{equation}{Gleichung}{Gleichungen}
% ****************************************************************************************

% ****************************************************************************************
% Bibliography settings
% ****************************************************************************************
% IEEEtran BibTeX style downloaded from: https://www.ctan.org/pkg/ieeetran
\bibliographystyle{IEEEtran} % choose the reference style
%
% ****************************************************************************************
% Glossary (acronyms, list of symbols) settings
% ****************************************************************************************
\usepackage[acronym,nomain,nonumberlist,nopostdot,sort=def,toc]{glossaries}
\renewcommand*{\glstextformat}[1]{\textcolor{black}{#1}} % make links appear black
\newglossary[slg]{symbolslist}{syi}{syg}{List of Symbols} % define custom glossary
\glsaddkey% define custom key
	{unit}% key
	{\glsentrytext{\glslabel}}% default value
	{\glsentryunit}% command analogous to \glsentrytext
	{\GLsentryunit}% command analogous to \Glsentrytext
	{\glsunit}% command analogous to \glstext
	{\Glsunit}% command analogous to \Glstext
	{\GLSunit}% command analogous to \GLStext
\glssetnoexpandfield{unit}
\makeglossaries % create makeindex files
% glossary of symbols is formatted as a longtable with three columns
\newglossarystyle{symbolsliststyle}{%
	\setglossarystyle{long3col}% style based on long3col
	\renewenvironment{theglossary}{%
		\begin{longtable}{lp{\glsdescwidth}>{\arraybackslash}p{2cm}}}%
		{\end{longtable}}%
	\renewcommand*{\glossaryheader}{% change the table header
		\bfseries Symbol & \bfseries Beschreibung & \bfseries Einheit\\\midrule%
		\endhead%
	}%
	\renewcommand*{\glossentry}[2]{% change the displayed items
		\glstarget{##1}{\glossentryname{##1}}% name
		& \glossentrydesc{##1}% description
		& $\glsentryunit{##1}$% unit
		\tabularnewline%
	}%
}


% ****************************************************************************************
% Page layout and headers
% ****************************************************************************************
\usepackage[
    includeheadfoot,% includes the head of the page into total body
	ignoremp,% disregards marginal notes in determining the horizontal margins
	nomarginpar,% shrinks spaces for marginal notes to 0pt
	hmargin=1.25in,% left and right margin
	vmargin=1in,% top and bottom margin
	headheight=14pt%  height of header
]{geometry} % specify page layout (paper name and orientation specified in doc class)
%\usepackage{parskip} % helps in implementing paragraph layouts
\KOMAoptions{parskip=half} % parskip=off, % parskip=half

\usepackage{fancyhdr} % header and footer settings
\pagestyle{fancy} % set page style to 'fancy'
\renewcommand{\chaptermark}[1]{\markboth{\thechapter.\ #1}{}}
\fancyhf{} % clear all header and footer fields
\fancyhead[L]{\nouppercase\leftmark} % set left header location (chapter)
\fancyfoot[C]{\thepage} % set center footer location (page count)


% ****************************************************************************************
% Matlab Code
% ****************************************************************************************
\usepackage{listings}
% MATLAB Code
\lstset
{language=Matlab,%
	basicstyle=\footnotesize,
	breaklines=true,%
	captionpos=b,
	frame = single,
	morekeywords={matlab2tikz},
	keywordstyle=\color{blue},%
	morekeywords=[2]{1}, keywordstyle=[2]{\color{black}},
	identifierstyle=\color{black},%
	stringstyle=\color{mylilas},
	commentstyle=\color{mygreen},%
	showstringspaces=false,%
	%numbers=left,%
	%numberstyle={\footnotesize \color{black}},%
	%stepnumber=5, %
	%numbersep=9pt, % 
	emph=[1]{for,end,break},emphstyle=[1]\color{red}, %
	emph=[2]{all}, emphstyle=[2]\color{mylilas},    
}

% ****************************************************************************************

\usepackage{lipsum}

% %% -> Layout Überprüfung
% \usepackage{layout}
% \usepackage{xspace}

% %% -> Einrücken von und Abstand zwischen Absätzen
% \setlength{\parindent}{0em}
% \setlength{\parskip}{1.5ex plus0.5ex minus0.5ex}

% %% -> Tabulator Funktion
% \newcommand\tab[1][1cm]{\hspace*{#1}}

% %% -> Weniger Warnungen wegen überfüllter Boxen
% \tolerance = 9999
% \sloppy



%% --------------------------------------------------------------------- %%

%% -> Bildnummerierung ist logisch mit Kapitelnummerierung
% \numberwithin{figure}{chapter}
% \numberwithin{table}{chapter}
% \numberwithin{equation}{chapter}
%\numberwithin{equation}{subsection}
%% --------------------------------------------------------------------- %%



%% --------------------------------------------------------------------- %%

%% -> Für das Symbolverzeichnis
% \usepackage[toc, hyperfirst = false, acronym, nonumberlist]{glossaries}
% \makeglossaries
%\input{Symbols_Acronym/symbols}
%\input{Symbols_Acronym/acronym}
%% --------------------------------------------------------------------- %%

% Boolesche Variable f"ur Bachelor-/Masterarbeit oder Bericht
\newboolean{thesis}