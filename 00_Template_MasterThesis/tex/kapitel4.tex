\chapter[Formatierungen]{Formeln}\label{cha-formeln}

Ein besonderer Vorteil von \LaTeX ist die schnelle und einfache Art Formeln einzugeben. Mit ein wenig "Ubung in der Nomenklatur gehen die komplexesten Ausdr"ucke problemlos von der Hand. Eine einfache Formel sieht folgendermaßen.
\begin{equation}
p_1+\frac{\rho v_1^2}{2}+\rho gh_1=p_2+\frac{\rho v_2^2}{2}+\rho gh_2+\Delta p.
\label{eqn-bernoulli}
\end{equation}
Oft ziehen sich Formeln "uber mehrere Zeilen  
\begin{eqnarray}
\Delta L&=&\int\limits_0^L(1-\cos\varphi)\,dx\approx\int\limits_0^L[1-(1-\varphi^2/2)]\,dx=\frac{1}{2}\int\limits_0^Lw'^2\,dx=\nonumber\\
&=&\frac{B^2\lambda^2}{2}\int\limits_0^L\cos^2\lambda x\,dx=\frac{B^2\lambda^2}{2}\left[\frac{\lambda x-\sin\lambda x\cos\lambda x}{2\lambda}\right]_0^L\approx\frac{B^2\lambda^2L}{4}
\end{eqnarray}
oder sind sehr kompliziert
\begin{eqnarray}
\boldsymbol{\tau}&=&2\mu\mathbf{D}=\mu[\nabla\vec{v}+(\nabla\vec{v})^T]\\
\boldsymbol{\sigma}'&=&\mu'\nabla\cdot\,\vec{v}\,\mathbf{I}=-\frac{2}{3}\mu\,\nabla\cdot(\nabla{v})\,\mathbf{I}.
\end{eqnarray}

