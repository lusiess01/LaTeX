\chapter{Introduction}
\label{chapter:introduction}
\section{Sample text}
\label{section:sample_text}
\lipsum[1]
\section{Another sample text}
\label{section:another_sample_text}
\lipsum[2]
\chapter{Theoretical background}
\label{chapter:theoretical_bg}
\section{Some facts}
\label{section:some_facts}
\lipsum[1]
\section{Some more facts}
\label{section:some_more_facts}
\lipsum[2]
\chapter{Tips and Tricks}
\label{chapter:tips_and_tricks}
\section{Cross-referencing and citing}
\label{section:cross_ref_and_citing}

\begin{itemize}
	\item \num{1234,56}  % Zahl mit Dezimaltrennzeichen
	\item \SI{2}{\meter^{-1}} % Einheit m^-1 ohne Bruch
	\item \SI{1.23e4}{\newton\meter} % Exponenten mit Punkt als Produkt
	\item \SI{1000}{\kilo\ohm} % Tausender ohne Trennzeichen
\end{itemize}
\section{Circuits and graphs}
\label{section:circuits_and_graphs}
\Cref{fig:demo_circuit} shows a simple linear and time-independent circuit, which is created with the package \texttt{circuitikz}. When using the \texttt{tikzexternalize} feature, the \texttt{circuitikz} environment must be substituted with \texttt{tikzpicture}. 
\begin{figure}[htbp]
	\centering
	\begin{tikzpicture}[scale=1,transform shape,european inductors,european resistors,
    american voltages]
    % grid definition
    % \draw[step=0.5,black!20,thin] (-4,-2.5) grid (4,2.5);
    % circuit definition
    \draw (0,0) coordinate (A) to [R,l=$R_\mathrm{1}$] ++ (-3,0) coordinate (B);
    \draw (B) to [vsource,l=$V_\mathrm{1}$] ++ (0,-2) -- ++ (3,0) coordinate (C);
    \draw (C) to [R,l=$R_\mathrm{L}$] (A);
    \draw (A) to [R,l_=$R_\mathrm{2}$,*-] ++ (3,0) coordinate (D);
    \draw (D) to [vsource,l=$V_\mathrm{2}$] ++ (0,-2) to [short,-*] (C);
\end{tikzpicture}
% EOF
	\caption{A simple circuit}
	\label{fig:demo_circuit}
\end{figure}

Waveforms captured by an oscilloscope are depicte \cite{deutsch}, im Anhang ist dann ein weiterer
Code verknüpft siehe dazu \ref{lst:test_code} und in \cref{fig:demo_circuit} und \ref{fig:demo_circuit} ist ein Schaltkreis mit tikz gezeichnet.
