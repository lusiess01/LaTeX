\chapter[Formatierungen]{Formatierungen von "Uberschriften und Text in Latex}

In Latex brauchen Sie sich um Formatierungen im Prinzip nicht k"ummern. Es ist lediglich notwendig, dass Sie Kapitel, Abschnitte, Unterabschnitte und so weiter als solche deklarieren.

Multiple Leerzeichen        werden von Latex einfach gel"oscht. Haben Sie einen Absatz beendet (nach 3 bis 4 S"atzen), dann lassen Sie durch ein zweimaliges ''Enter'' eine Zeile Abstand. Der Absatz wird je nach globaler Einstellung einger"uckt oder abgesetzt.

Wollen Sie im Text etwas hervorheben, dann verwenden Sie \emph{hervorgehoben}. Die Hervorhebung wird von Latex automatisch dem jeweiligen Textstil angepasst. Sie k"onnen aber auch etwas explizit \textbf{fett}, \textit{kursiv} oder \underline{unterstrichen} setzen, wobei dies mit Vorsicht zu genie"sen ist.

\section[Abschnitt]{Das wäre ein Abschnitt}

Mit etwas Text ...

\subsection[Unterabschnitt]{Bzw. ein Unterabschnitt}

Wie Ihnen vielleicht schon aufgefallen ist, vergr"o"sert \LaTeX nach einem ''.'' den Abstand geb"uhrlich f"ur ein Satzende. Falls dies nicht ben"otigt wird z.B.~hier, sollte dies h"andisch verhindert werden.

\paragraph{Gliederungsebene 3} Die n"achste Gliederungsebene wird nicht mehr nummeriert.

\LaTeX kennt auch Aufz"ahlungen wobei es diese mit
\begin{enumerate}
\item Nummerierung
\item\label{enum-ebene} auf der ersten Ebene
\item oder
\begin{itemize}
\item ohne Nummerierung
\item auf der 2.~Ebene gibt.
\end{itemize}
\end{enumerate}
Es gibt eine ganze Reihe von weiteren Formatierungsm"oglichkeiten. Z.B.~behandelt {\LaTeX} die erste Seite eines Kapitels anders als alle folgenden. Dies f"allt insbesondere bei der Seitenzahl und der Kopfzeile auf.
